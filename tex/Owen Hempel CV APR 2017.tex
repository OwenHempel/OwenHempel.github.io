\documentclass{my_cv}
\usepackage{nopageno}

\title{\bfseries\Huge Owen Hempel, P. Eng}
\author{owenhempel@gmail.com}
\date{}

\begin{document}

\maketitle

\vspace{1em}
\begin{minipage}[ht]{0.48\textwidth}
356 Parkwood Drive

Port Elgin, Ontario
\end{minipage}
\begin{minipage}[ht]{0.48\textwidth}
Cell: 226-668-8118
\end{minipage}


\section{Education}
\datedsubsection{McMaster University}{2011--2016}
I completed a Bachelor of Engineering in Mechatronics Engineering at McMaster University. Through this program I learned a combination of Mechanical, Electrical, and Software Engineering fundamentals, which have served me well in my roles at Bruce Power.

\section{Work}
\datedsubsection{Responsible System Engineer, Nuclear Systems}{March 2021--Present}
When the Engineering organization re-organized a second time, I was transferred back to a systems engienering role, this time in the Nuclear Systems section. I have been assigned to Shutdown System \#1


\datedsubsection{Equipment Performance Engineer, Instrumentation \& Control Department}{September 2020--March 2021}
When the Engineering organization re-organized to a new structure, I was transferred from Fuel Handling Engineering to Instrumentation \& Control, where I held a portfolio of the Primary Heat Transport and Moderator systems, as well as their respective Auxiliaries. This role gave me a significantly better understanding of the process systems outside the fuel handling realm. 

\datedsubsection{Equipment Performance Engineer, Fuel Handling Department}{January 2019--September 2020}

In January of 2019, our organization shifted to work more closely with Maintenance Teams. My regular duties did not change from my role as a Responsible System Engineer, however the focus on maintenance has allowed me to gain more perspective on the short term execution of work, while allowing me to retain strategy-level involvement as well. My system portfolio in this role was unchanged from my previous Fuel Handling station engineering responsibilities. In this role I provided shift-based support during the MCR6 defuel evolution, which came in nearly two weeks ahead of schedule.

\datedsubsection{Engineering Outage Manager, Outages (B1881)}{September 2018--November 2018}
As EOM for B1881, I was assigned a rotating shift crew, and was responsible for all Outage issues and communication between the various Outage stakeholders and Plant Engineering departments. This role involved a large volume of communication and required strong organizational skills, communication skills, and rapid issue triage \& response. This was my first experience in a management role, and although it was temporary, I found the work engaging. Of note, critical decision making skills and my previous experience in the high-stress type of situations prepared me well and allowed me to perform strongly.

\datedsubsection{Bruce Power, Plant Engineering - Fuel Handling}{June 2016--January 2019}

In this role, I was responsible for equipment health. My section dealt with the equipment that fuels the reactors while they are operating, and performs various maintenance support functions during outages. The focus for this equipment is as close to 100% reliability as possible while maintaining financial viability.

My responsibilities ranged from evaluations for safe operation, to creating and implementing preventive maintenance plans. Troubleshooting, performance monitoring and reporting, and project development were all activities that station engineers are expected to perform.

This role frequently involved working with small groups or crews, typically on shifts, to execute work. This means clear communication, good turnover, and strong technical knowledge foundations are required. Engineers are expected to advocate from the perspective of equipment health, and to always keep safe operation as the highest priority.

My system portfolios at various times within the section included Fuel Handling Computers, Inverters, Power Tracks, and Power Distribution, and the water circulation systems.



\end{document} 